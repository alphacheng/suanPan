\documentclass[11pt,fleqn,3p]{elsarticle}
\usepackage{amsmath,amsfonts,amssymb,mathpazo,indentfirst}
\newcommand*{\md}[1]{\mathrm{d}#1}
\newcommand*{\mT}{\mathrm{T}}
\newcommand*{\tr}[1]{\mathrm{tr}~#1}
\newcommand*{\dev}[1]{\mathrm{dev}~#1}
\newcommand*{\ddfrac}[2]{\dfrac{\md#1}{\md#2}}
\newcommand*{\pfrac}[2]{\dfrac{\partial#1}{\partial#2}}
\title{SimpleSand Model}\date{}\author{tlc}
\begin{document}\pagestyle{empty}
\section{Yield Function}
A wedge-like function is chosen to be the yield surface.
\begin{gather*}
F=\big|\mathbold{s}+p\mathbold{\alpha}\big|+mp=\big|\mathbold{\eta}\big|+mp,
\end{gather*}
where $\mathbold{s}=\dev{\mathbold{\sigma}}$ is the deviatoric stress, $p=\dfrac{1}{3}\tr{\mathbold{\sigma}}$ is the hydrostatic stress, $\alpha$ is the so called back stress ratio and $m$ characterises the size of the wedge. For simplicity, $m$ is assumed to be a constant in this model.
\section{Flow Rule}
A non-associated plastic flow is used, the corresponding flow rule is defined as follows.
\begin{gather*}
\Delta\mathbold{\varepsilon}^p=\gamma\left(\mathbold{n}+\dfrac{1}{3}A\left(\alpha^d-\mathbold{\alpha}:\mathbold{n}\right)\mathbold{I}\right),
\end{gather*}
where $D=A\left(\alpha^d-\mathbold{\alpha}:\mathbold{n}\right)$ is the dilatancy parameter, $A$ is a model constant, it can also be defined as a function of current state.

Again for simplicity, the linear elasticity is applied, hence the following expressions hold.
\begin{gather*}
\mathbold{s}=\mathbold{s}^{tr}-2G\gamma\mathbold{n},\qquad
p=p^{tr}-KA\gamma\left(\alpha^d-\mathbold{\alpha}:\mathbold{n}\right).
\end{gather*}
A hyper-elastic response can also be applied. However, in that case, the $G$ and $K$ would be zero for zero hydrostatic pressure.
\section{Hardening Rule}
The evolution rate of the back stress ratio $\mathbold{\alpha}$ is defined in terms of a proper distance measure from the bounding surface, Here, such a distance measure is chosen to be $\alpha^b\mathbold{n}-h\mathbold{\alpha}$, where $h$ is a model constant. Thus,
\begin{gather*}
\Delta\mathbold{\alpha}=\gamma{}\left(\alpha^b\mathbold{n}-h\mathbold{\alpha}\right).
\end{gather*}
Essentially, this is similar to the Armstrong--Frederick type kinematic hardening rule.
\section{Critical State}
The state parameter is defined to be
\begin{gather*}
\psi=v_0\left(1+\tr{\mathbold{\varepsilon}^{tr}}\right)-v_c+\lambda_c\ln\left(\dfrac{p}{p_c}\right),
\end{gather*}
where $v_0$ is the initial specific volume, $v_c$ is the corresponding specific volume on the critical line at $p_c$ and $\lambda_c$ is absolute value of the slope of the critical line in $v$--$\ln(-p)$ space.

The Lode angle dependence is removed for simple derivations of the corresponding terms. Hence both the dilatancy surface and bounding surface will be circular cones in the principal stress space.

The dilatancy surface is defined as
\begin{gather*}
\alpha^d=\alpha^c\exp\left(n^d\psi\right).
\end{gather*}
The corresponding derivatives are
\begin{gather*}
\pfrac{\alpha^d}{\mathbold{\varepsilon}^{tr}}=\alpha^dn^dv_0\mathbold{I},\qquad
\pfrac{\alpha^d}{p}=\alpha^dn^d\dfrac{\lambda_c}{p}.
\end{gather*}

The bounding surface is defined as
\begin{gather*}
\alpha^b=\alpha^c\exp\left(-n^b\psi\right).
\end{gather*}
The corresponding derivatives are
\begin{gather*}
\pfrac{\alpha^b}{\mathbold{\varepsilon}^{tr}}=-\alpha^bn^bv_0\mathbold{I},\qquad
\pfrac{\alpha^b}{p}=-\alpha^bn^b\dfrac{\lambda_c}{p}.
\end{gather*}

The symbol $\mathbold{I}$ denotes the unit second order tensor.
\section{Local Iteration}
According to tensor algebra, the following expressions can be derived.
\begin{gather*}
\pfrac{|\mathbold{\eta}|}{}=\mathbold{n}:\pfrac{\mathbold{\eta}}{},\qquad
\pfrac{\mathbold{n}}{}=\dfrac{1}{|\mathbold{\eta}|}\left(\pfrac{\mathbold{\eta}}{}-\mathbold{n}\otimes\left(\mathbold{n}:\pfrac{\mathbold{\eta}}{}\right)\right).
\end{gather*}
Hence,
\begin{gather*}
\pfrac{|\mathbold{\eta}|}{p}=\mathbold{n}:\mathbold{\alpha},\qquad
\pfrac{\mathbold{n}}{p}=\dfrac{1}{|\mathbold{\eta}|}\left(\mathbold{\alpha}-\left(\mathbold{n}:\mathbold{\alpha}\right)\mathbold{n}\right),\\
\pfrac{|\mathbold{\eta}|}{\mathbold{s}}=\mathbold{n}:\mathbf{I},\qquad
\pfrac{\mathbold{n}}{\mathbold{s}}=\dfrac{1}{|\mathbold{\eta}|}\left(\mathbf{I}-\mathbold{n}\otimes\left(\mathbold{n}:\mathbf{I}\right)\right),\\
\pfrac{|\mathbold{\eta}|}{\mathbold{\alpha}}=p\mathbold{n}:\mathbf{I},\qquad
\pfrac{\mathbold{n}}{\mathbold{\alpha}}=\dfrac{p}{|\mathbold{\eta}|}\left(\mathbf{I}-\mathbold{n}\otimes\left(\mathbold{n}:\mathbf{I}\right)\right).
\end{gather*}
In summary, there are four residual equations.
\begin{gather*}
\mathbold{R}=\left\{\begin{array}{l}
|\mathbold{\eta}|+mp,\\[3mm]
p-p^{tr}+KA\gamma\left(\alpha^d-\mathbold{\alpha}:\mathbold{n}\right),\\[3mm]
\mathbold{s}-\mathbold{s}^{tr}+2G\gamma\mathbold{n},\\[3mm]
\mathbold{\alpha}_n+\gamma{}\alpha^b\mathbold{n}-\left(\gamma{}h+1\right)\mathbold{\alpha}.
\end{array}\right.
\end{gather*}

By defined $\mathbold{x}=\begin{bmatrix}\gamma&p&\mathbold{s}&\mathbold{\alpha}\end{bmatrix}^\mT$, the local Jacobian can be derived as
\begin{small}
\begin{gather*}
\pfrac{\mathbold{R}}{\mathbold{x}}=
\begin{bmatrix}
	0                                                      & \mathbold{\alpha}:\mathbold{n}+m                                                     & \mathbold{n}:\mathbf{I}                                       & p\mathbold{n}:\mathbf{I}                                                                                \\[4mm]
	KA\left(\alpha^d-\mathbold{\alpha}:\mathbold{n}\right) & 1+KA\gamma\left(\pfrac{\alpha^d}{p}-\mathbold{\alpha}:\pfrac{\mathbold{n}}{p}\right) & -KA\gamma\mathbold{\alpha}:\pfrac{\mathbold{n}}{\mathbold{s}} & -KA\gamma\left(\mathbf{I}:\mathbold{n}+\mathbold{\alpha}:\pfrac{\mathbold{n}}{\mathbold{\alpha}}\right) \\[4mm]
	2G\mathbold{n}                                         & 2G\gamma\pfrac{\mathbold{n}}{p}                                                      & \mathbf{I}+2G\gamma\pfrac{\mathbold{n}}{\mathbold{s}}         & 2G\gamma\pfrac{\mathbold{n}}{\mathbold{\alpha}}                                                         \\[4mm]
	\alpha^b\mathbold{n}-h\mathbold{\alpha}                & \gamma\left(\pfrac{\alpha^b}{p}\mathbold{n}+\alpha^b\pfrac{\mathbold{n}}{p}\right)   & \gamma\alpha^b\pfrac{\mathbold{n}}{\mathbold{s}}              & \gamma\alpha^b\pfrac{\mathbold{n}}{\mathbold{\alpha}}-\left(\gamma{}h+1\right)\mathbf{I}
\end{bmatrix}.
\end{gather*}
\end{small}
In the explicit form, it is
\begin{scriptsize}
\begin{gather*}
\begin{bmatrix}
	0                                                      & \mathbold{\alpha}:\mathbold{n}+m                                                                                                                                                  & \mathbold{n}:\mathbf{I}                                                                                                                   & p\mathbold{n}:\mathbf{I}                                                                                                                                             \\[4mm]
	KA\left(\alpha^d-\mathbold{\alpha}:\mathbold{n}\right) & 1+KA\gamma\left(\pfrac{\alpha^d}{p}-\dfrac{\mathbold{\alpha}:\mathbold{\alpha}-\left(\mathbold{n}:\mathbold{\alpha}\right)^2}{|\mathbold{\eta}|}\right)                           & \dfrac{KA\gamma}{|\mathbold{\eta}|}\left(\left(\mathbold{\alpha}:\mathbold{n}\right)\cdot\mathbold{n}-\mathbold{\alpha}\right):\mathbf{I} & KA\gamma\left(\mathbold{n}+\dfrac{p}{|\mathbold{\eta}|}\left(\left(\mathbold{\alpha}:\mathbold{n}\right)\cdot\mathbold{n}-\mathbold{\alpha}\right)\right):\mathbf{I} \\[4mm]
	2G\mathbold{n}                                         & \dfrac{2G\gamma}{|\mathbold{\eta}|}\left(\mathbold{\alpha}-\left(\mathbold{n}:\mathbold{\alpha}\right)\cdot\mathbold{n}\right)                                                    & \mathbf{I}+\dfrac{2G\gamma}{|\mathbold{\eta}|}\left(\mathbf{I}-\mathbold{n}\otimes\left(\mathbold{n}:\mathbf{I}\right)\right)             & \dfrac{2G\gamma{}p}{|\mathbold{\eta}|}\left(\mathbf{I}-\mathbold{n}\otimes\left(\mathbold{n}:\mathbf{I}\right)\right)                                                \\[4mm]
	\alpha^b\mathbold{n}-h\mathbold{\alpha}                & \gamma\left(\pfrac{\alpha^b}{p}\mathbold{n}+\dfrac{\alpha^b}{|\mathbold{\eta}|}\left(\mathbold{\alpha}-\left(\mathbold{n}:\mathbold{\alpha}\right)\cdot\mathbold{n}\right)\right) & \dfrac{\gamma\alpha^b}{|\mathbold{\eta}|}\left(\mathbf{I}-\mathbold{n}\otimes\left(\mathbold{n}:\mathbf{I}\right)\right)                  & \dfrac{\gamma\alpha^bp}{|\mathbold{\eta}|}\left(\mathbf{I}-\mathbold{n}\otimes\left(\mathbold{n}:\mathbf{I}\right)\right)-\left(\gamma{}h+1\right)\mathbf{I}
\end{bmatrix}.
\end{gather*}
\end{scriptsize}
\section{Consistent Tangent Stiffness}
Accordingly,
\begin{gather*}
\pfrac{\mathbold{R}}{\mathbold{\varepsilon}^{tr}}=
\begin{bmatrix}
	0                                                    \\[3mm]
	\left(KA\gamma\alpha^dn^dv_0-K\right)\mathbold{I}    \\[3mm]
	-2G\mathbf{I}_{d}                                    \\[3mm]
	-\alpha^bn^bv_0\gamma\mathbold{n}\otimes\mathbold{I}
\end{bmatrix}.
\end{gather*}
Thus,
\begin{gather*}
\ddfrac{\mathbold{x}}{\mathbold{\varepsilon}^{tr}}=-\left(\pfrac{\mathbold{R}}{\mathbold{x}}\right)^{-1}\pfrac{\mathbold{R}}{\mathbold{\varepsilon}^{tr}}.
\end{gather*}

The stress update is
\begin{gather*}
\mathbold{\sigma}=\mathbold{s}+p\mathbold{I}.
\end{gather*}
The consistent tangent stiffness is
\begin{gather*}
\ddfrac{\mathbold{\sigma}}{\mathbold{\varepsilon}^{tr}}=\ddfrac{\mathbold{s}}{\mathbold{\varepsilon}^{tr}}+\mathbold{I}\otimes\ddfrac{p}{\mathbold{\varepsilon}^{tr}}.
\end{gather*}
\end{document}
