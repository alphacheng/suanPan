\documentclass[a4paper,10pt,fleqn]{article}
\usepackage[margin=20mm]{geometry}
\usepackage{mathpazo,amsmath,amsfonts,amssymb}
\usepackage[colorlinks]{hyperref}
\newcommand*{\md}[1]{\mathrm{d}#1}
\newcommand*{\pfrac}[2]{\dfrac{\partial#1}{\partial#2}}
\newcommand*{\ddfrac}[2]{\dfrac{\md#1}{\md#2}}
\title{VAFCRP Model}
\author{Theodore Chang}
\date{}\pagestyle{empty}
\begin{document}
\noindent{}This document introduces the implementation of the von Mises based model that employs a Voce (1955) type isotropic hardening rule, a Armstrong--Frederick (Armstrong and Frederick, 1966) type kinematic hardening rule that supports multiplicative back stresses as proposed by Chaboche and Rousselier (1983) and a Peric (1993) type viscosity rule.

\noindent{}References:
\begin{enumerate}
\item \href{https://doi.org/10.1017/S0368393100118759}{https://doi.org/10.1017/S0368393100118759}
\item \href{https://doi.org/10.1179/096034007X207589}{https://doi.org/10.1179/096034007X207589}
\item \href{https://doi.org/10.1016/0749-6419(89)90015-6}{https://doi.org/10.1016/0749-6419(89)90015-6}
\item \href{https://doi.org/10.1002/nme.1620360807}{https://doi.org/10.1002/nme.1620360807}
\end{enumerate}
\section{Yield Function}
A von Mises yielding function is used.
\begin{gather}
F=\sqrt{\dfrac{3}{2}}\Big|\eta\Big|-k=q-k,
\end{gather}
in which $\eta=s-\beta$ is the shifted stress, $s$ is the stress deviator, $\beta$ is the back stress and $k$ is the isotropic hardening stress.
\section{Flow Rule}
The associated plasticity flow is adopted. The plastic strain rate is then
\begin{gather}
\md{\varepsilon^p}=\gamma\pfrac{F}{\sigma}=\sqrt{\dfrac{3}{2}}\gamma{}n,
\end{gather}
where $n=\dfrac{\eta}{\Big|\eta\Big|}$. The corresponding accumulated plastic strain rate is
\begin{gather}
\md{p}=\sqrt{\dfrac{2}{3}\md{\varepsilon^p}:\md{\varepsilon^p}}=\gamma.
\end{gather}
\section{Plastic Multiplier}
The rate of plastic multiplier is defined as
\begin{gather}
\dfrac{\gamma}{\Delta{}t}=\dot{\gamma}=\dfrac{1}{\mu}\left(\left(\dfrac{q}{k}\right)^{\dfrac{1}{\epsilon}}-1\right),
\end{gather}
in which $\mu$ and $\epsilon$ are two material constants. Equivalently, it is
\begin{gather}
q\left(\dfrac{\Delta{}t}{\Delta{}t+\mu\gamma}\right)^\epsilon-k=0.
\end{gather}
\section{Hardening Rules}
An exponential function with a linear component is used for isotropic hardening stress.
\begin{gather}
k=\sigma_y+k_lp+k_s-k_se^{-mp}.
\end{gather}
The corresponding derivative is
\begin{gather}
\ddfrac{k}{\gamma}=k_l+k_sme^{-mp}.
\end{gather}
The rate form of back stress $\displaystyle\beta=\sum\beta^i$ is defined as
\begin{gather*}
\md{\beta^i}=\sqrt{\dfrac{2}{3}}a^i\md{\varepsilon^p}-b^i\beta^i\md{p}.
\end{gather*}
In terms of $\gamma$, it is $\md{\beta^i}=a^i\gamma{}n-b^i\gamma\beta^i$. The incremental form is thus
\begin{gather}
\beta^i=\beta_n^i+a^i\gamma{}n-b^i\gamma\beta^i,\qquad
\beta^i=\dfrac{\beta_n^i+a^i\gamma{}n}{1+b^i\gamma}.
\end{gather}
\section{Incremental Form}
The shifted stress can be computed as
\begin{gather}
\eta=s-\beta=2GI_d\left(\varepsilon^{tr}-\varepsilon^p_n-\sqrt{\dfrac{3}{2}}\gamma{}n\right)-\beta=s^{tr}-\sqrt{6}G\gamma{}n-\sum\dfrac{\beta_n^i+a^i\gamma{}n}{1+b^i\gamma}
\end{gather}
with $s^{tr}=2GI_d\left(\varepsilon^{tr}-\varepsilon^p_n\right)$. Hence,
\begin{gather*}
\Big|\eta\Big|n+\sqrt{6}G\gamma{}n+\sum\dfrac{a^i\gamma}{1+b^i\gamma}n=s^{tr}-\sum\dfrac{\beta_n^i}{1+b^i\gamma},\qquad
\Big|\eta\Big|+\sqrt{6}G\gamma+\sum\dfrac{a^i\gamma}{1+b^i\gamma}=\Big|s^{tr}-\sum\dfrac{\beta_n^i}{1+b^i\gamma}\Big|.
\end{gather*}
Eventually,
\begin{gather}
\Big|\eta\Big|=\Big|s^{tr}-\sum\dfrac{\beta_n^i}{1+b^i\gamma}\Big|-\sqrt{6}G\gamma-\sum\dfrac{a^i\gamma}{1+b^i\gamma}.
\end{gather}
Then $\eta$ can be expressed as,
\begin{gather*}
\eta=\dfrac{\Big|s^{tr}-\sum\dfrac{\beta_n^i}{1+b^i\gamma}\Big|-\sqrt{6}G\gamma-\sum\dfrac{a^i\gamma}{1+b^i\gamma}}{\Big|s^{tr}-\sum\dfrac{\beta_n^i}{1+b^i\gamma}\Big|}\left(s^{tr}-\sum\dfrac{\beta_n^i}{1+b^i\gamma}\right).
\end{gather*}
It is equivalent to
\begin{gather*}
\eta=\left(\Big|s^{tr}-\sum\dfrac{\beta_n^i}{1+b^i\gamma}\Big|-\sqrt{6}G\gamma-\sum\dfrac{a^i\gamma}{1+b^i\gamma}\right)u.
\end{gather*}
It is easy to see that $n=u$ with $\displaystyle{}u=\dfrac{s^{tr}-\sum\dfrac{\beta_n^i}{1+b^i\gamma}}{\Big|s^{tr}-\sum\dfrac{\beta_n^i}{1+b^i\gamma}\Big|}$. The derivatives of $u$ are
\begin{gather}
\pfrac{u}{\gamma}=\dfrac{1}{\Big|s^{tr}-\sum\dfrac{\beta_n^i}{1+b^i\gamma}\Big|}\sum\dfrac{b^i}{(1+b^i\gamma)^2}\left(\beta_n^i-u:\beta_n^iu\right),\qquad
\pfrac{u}{\varepsilon^{tr}}=\dfrac{1}{\Big|s^{tr}-\sum\dfrac{\beta_n^i}{1+b^i\gamma}\Big|}2G\left(I_d-u\otimes{}u\right).
\end{gather}
Furthermore,
\begin{gather}
\pfrac{q}{\gamma}=\sqrt{\dfrac{3}{2}}\sum\dfrac{b^iu:\beta_n^i-a^i}{(1+b^i\gamma)^2}-3G.
\end{gather}
\section{Scalar Equation Iteration}
For viscoplasticity, the yield function $F$ is not necessarily zero. The rate form of plastic multiplier is used here for iteration.
\begin{gather*}
R=q\left(\dfrac{\Delta{}t}{\Delta{}t+\mu\gamma}\right)^\epsilon-k=0.
\end{gather*}
The corresponding derivatives are then
\begin{gather}
\pfrac{R}{\gamma}=\left(\dfrac{\Delta{}t}{\Delta{}t+\mu\gamma}\right)^\epsilon\left(\pfrac{q}{\gamma}-\dfrac{q\epsilon\mu}{\Delta{}t+\mu\gamma}\right)-\ddfrac{k}{\gamma},\\
\pfrac{R}{\varepsilon^{tr}}=\left(\dfrac{\Delta{}t}{\Delta{}t+\mu\gamma}\right)^\epsilon\sqrt{6}Gu:I_d=\left(\dfrac{\Delta{}t}{\Delta{}t+\mu\gamma}\right)^\epsilon\sqrt{6}Gu,
\end{gather}
with
\begin{gather}
k=\sigma_y+k_l\left(p_n+\gamma\right)+k_s\left(1-e^{-m\left(p_n+\gamma\right)}\right),\\
q=\sqrt{\dfrac{3}{2}}\left(\Big|s^{tr}-\sum\dfrac{\beta_n^i}{1+b^i\gamma}\Big|-\sqrt{6}G\gamma-\sum\dfrac{a^i\gamma}{1+b^i\gamma}\right).
\end{gather}
\section{Consistent Tangent Stiffness}
For stiffness, $\varepsilon^{tr}$ is now varying, then
\begin{gather}
\pfrac{R}{\varepsilon^{tr}}+\pfrac{R}{\gamma}\ddfrac{\gamma}{\varepsilon^{tr}}=0,\qquad\ddfrac{\gamma}{\varepsilon^{tr}}=-\left(\pfrac{R}{\gamma}\right)^{-1}\pfrac{R}{\varepsilon^{tr}}.
\end{gather}
Since the stress can be written as
\begin{gather}
\sigma=E(\varepsilon^{tr}-\varepsilon^p)=E(\varepsilon^{tr}-\varepsilon^p_n-\Delta\varepsilon^p)=E(\varepsilon^{tr}-\varepsilon^p_n)-\sqrt{6}G\gamma{}u.
\end{gather}
The derivative is
\begin{gather}
\begin{split}
\ddfrac{\sigma}{\varepsilon^{tr}}&=E-\sqrt{6}G\left(\gamma\pfrac{u}{\varepsilon^{tr}}+\left(u+\gamma\pfrac{u}{\gamma}\right)\ddfrac{\gamma}{\varepsilon^{tr}}\right)\\
&=E+\sqrt{6}G\left(\left(u+\gamma\pfrac{u}{\gamma}\right)\left(\pfrac{R}{\gamma}\right)^{-1}\pfrac{R}{\varepsilon^{tr}}-\gamma\pfrac{u}{\varepsilon^{tr}}\right).
\end{split}
\end{gather}
\end{document}
